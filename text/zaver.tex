\chapter*{Závěr}
\addcontentsline{toc}{chapter}{Závěr}

V~této práci jsem se pokusil navrhnout a~naprogramovat datovou strukturu, která
kombinuje výhody databází jako perzistenci a~transakce s~rychlostí běžných datových
struktur. Výsledkem práce je návrh
databáze, která udržuje data v~paměti, implementované jako knihovna
funkcí. Práce s~databází a~podpora transakcí je realizována pomocí konceptu
transakční paměti. Díky tomu je s~databází možné snadno pracovat paralelně z~více vláken.

Stále ale zbývá mnoho možností, jak současnou implementaci rozšířit či vylepšit.
Technicky nejjednodušší způsob zrychlení současné implementace je možnost přepsat
některé části, především výpočet CRC, s~pomocí SSE instrukcí a~tak je významně urychlit.
Mezi podstatně složitější, ale v~jistém směru užitečnější vylepšení patří implementace
mechanismu pro automatické restartování transakcí. 

Současná implementace sice obsahuje
makra \verb|trBegin| a~\verb|trCommit|, která automatické restartování umožňují,
ta ale používají
exponenciálně dlouhé čekání před dalším pokusem. To
je výhodné v~situaci, kdy jsou transakce malé a~množství kolizí nízké. Implementace
pokročilejšího algoritmu by nejspíše vyžadovala možnost nějak zvýhodnit konkrétní
transakce (především ty, které již opakovaně selhaly), což nyní není možné.

Dalším rozšířením, které by mohlo být velmi užitečné z~praktického hlediska, je zavedení
složených typů atributů jako jsou seznamy či slovníky (množiny párů klíč--hodnota
s~rychlým hledáním podle klíče). Hlavní obtíží v~tomto případě je návrh vhodného
rozhraní pro práci s~nimi a~serializaci daných operací na disk. Všechny současné atributy
si totiž vystačí s~operacemi \uv{přečti hodnotu} a~\uv{přiřaď hodnotu},
zatímco pro složené
atributy jsou tyto operace krajně nevhodné a~bylo by nutné zavést operace nové.

Posledním možným rozšířením, o~kterém se zmíním, jsou rozšířené kontexty indexů --
tedy že by kontext indexu nebyla pouze jeden objekt globální pro celou databázi, ale
indexy by mohly ukládat data i~do jednotlivých uzlů. To by mohlo zrychlit operace,
kdy je třeba uzel vyhledat v~rámci datové struktury představující index.

