\chapter*{Úvod}
\addcontentsline{toc}{chapter}{Úvod}

Tato práce se snaží vyplnit mezeru, která se rozevírá mezi (relačními) databázovými systémy
na straně jedné a~klasickými datovými strukturami na straně druhé. Dnešní
 databáze se zaměřují na zpracování obrovských množství dat, a~proto
užívají velmi sofistikované metody manipulace s~nimi. Tyto
pokročilé postupy mají však i~svou stinnou stránku -- nemalou časovou i~prostorovou
režii. Ta se u~obrovských souborů dat neprojevuje,
neboť je bohatě kompenzována vyšší efektivitou zpracování,
ale u~malých dat bývá významná.
Oproti tomu datové struktury jsou velmi rychlé a~mají flexibilnější rozhraní,
které lze často přizpůsobit konkrétní aplikaci, ale postrádají možnost perzistentně
ukládat data (tedy tak aby přežila restart či pád aplikace).

V~důsledku zvětšování velikosti operační paměti počítačů
stále častěji nastává situace, kdy se potřebná data vejdou do operační paměti.
Pokud je u~dat požadováno trvalé uložení mezi běhy aplikace a~jsou tak cenná,
že je nutné zajistit možnost jejich obnovy i~v~případě havárie systému,
nezbývá programátorovi, než místo rychlé datové struktury použít databázi.
Použití databáze zajistí potřebnou perzistenci a~konzistentnost dat, ale
oproti použití datové struktury negativně ovlivní výkon aplikace.

Z~toho důvodu jsem se rozhodl vytvořit hybridní databázi, která se snaží kombinovat
výhody databází a~datových struktur. Udržuje všechna data v~paměti, 
ve tvaru co nejpřátelštějším k~programovacímu
jazyku dané aplikace. Zajišťuje uložení jejich konzistentní verze na disku
a~umožňuje pracovat s~nimi pomocí transakcí. Podpora transakcí je důležitá
jak z~důvodu určení konzistentní verze dat pro uložení na disku, tak především
s~ohledem na stále narůstající počet (logických) procesorů v~dnešních
počítačích, neboť elegantně eliminuje potřebu složité ruční synchronizace přístupu
k~datům,
která je velmi častým zdrojem těžko odhalitelných chyb.

Součástí práce je i~ukázková implementace navržené databáze v~jazyce C ve~formě
knihovny funkcí. Jazyk C byl zvolen pro svou rozšířenost, a~to především v~unixovém
světě. Výhodou jazyka C oproti C{\tt++} je jeho rozšířenost a~\uv{kompatibilita} -- zatímco
užívat z~C{\tt++} knihovny vytvořené v~C je velmi snadné, je použití knihoven
vytvořených v~C{\tt++}, především obsahuje-li jejich rozhraní šablony, v~podstatě nemožné
(přičemž podobná situace nastává s~vazbami do jiných programovacích jazyků).

\subsection*{Rozčlenění práce}
\begin{itemize}
  \item První kapitola obsahuje motivaci k~vytvoření hybridní databáze a~hrubý
    popis jejího fungování.
  \item Druhá kapitola se zabývá softwarovou transakční pamětí a~její implementací
    použitou v~této práci.
  \item Třetí kapitola popisuje formát uložení databáze na disku.
  \item Čtvrtá kapitola obsahuje popis rozhraní navržené databáze.
  \item V~páté kapitole jsou rozebrány implementačně závislé aspekty hybridní
    databáze.
  \item Příloha 1 obsahuje příklad jednoduchého programu, který využívá hybridní databázi.
  \item V~příloze 2 naleznete výsledky testů škálovatelnosti.
\end{itemize}

